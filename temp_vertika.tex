\documentclass[a4paper,10pt]{article}
%-----------------------------------------------------------
% {{{

%for courses
\usepackage{changepage}

%\usepackage[top=0.75in, bottom=0.75in, left=0.55in, right=0.85in]{geometry}
\usepackage[top=0.75in, bottom=0.5in, left=0.5in, right=0.5in]{geometry}


\usepackage{graphicx}
\usepackage{hyperref}
\usepackage{url}
\usepackage{palatino}
\usepackage{booktabs}
\fontfamily{SansSerif}
\selectfont

\usepackage[T1]{fontenc}
\usepackage[ansinew]{inputenc}
% \usepackage{helvetica}
% \usepackage{array}
\usepackage{color}
\definecolor{lightgray}{rgb}{0.83, 0.83, 0.83}

\definecolor{gainsboro}{rgb}{0.86, 0.86, 0.86}
\usepackage{color, colortbl}
%Light gray:
\definecolor{Gray}{gray}{0.94}

\textheight=10in
\raggedbottom
% \raggedright
\setlength{\tabcolsep}{0in}
\newcommand{\isep}{-2 pt}
\newcommand{\sectsep}{-0.3cm}
\newcommand{\lsep}{-0.5cm}
\newcommand{\psep}{-0.5cm}
\newcommand{\hsep}{-0.6cm}
\newcommand{\bsep}{-0.55cm}
\renewcommand{\labelitemii}{$\circ$}
% Adjust margins
%\addtolength{\oddsidemargin}{-0.375in}
%\addtolength{\evensidemargin}{-0.375in}
%\addtolength{\textwidth}{1.75in}
%\addtolength{\topmargin}{0.1375in}
%\addtolength{\textheight}{1.75in}
\pagestyle{empty}
% }}}
%-----------------------------------------------------------
%Custom commands
\newcommand{\resitem}[1]{\item #1 \vspace{-2pt}}


% \newcommand{\resheading}[1]{{\small
%         {
%             \begin{minipage}
%                 {0.975\textwidth}\textbf{{\textsc{#1 \vphantom{p\^{E}} }}}
%                 \\[-0.3cm]
%                 \hrule
%             \end{minipage}
%             \\[-0.5cm]
%         }
%  }} 
 
 \newcommand{\resheading}[1]{{\small \colorbox{lightgray}{\begin{minipage}{0.975\textwidth}{\textbf{#1 \vphantom{p\^{E}}}}\end{minipage}}}}
 
\newcommand{\ressubheading}[3]{
\begin{tabular*}{6.62in}{l @{\extracolsep{\fill}} r}
    \textsc{{\textbf{#1}}} & \textsc{\textit{[#2]}} \\
\end{tabular*}\vspace{-8pt}}
% \textit{#3} &  \\
%-----------------------------------------------------------





\begin{document}

%{{{
\hspace{0.5cm}\\
\hspace{0.5cm}\\
\hspace{0.5cm}\\
\hspace{0.5cm}\\
\hspace{0.5cm}\\
\hspace{0.5cm}\\
\hspace{0.5cm}\\
\hspace{0.5cm}\\
\hspace{0.5cm}\\
\hspace{0.5cm}\\
\hspace{0.5cm}\\
\hspace{0.5cm}\\
\hspace{0.5cm}\\[-0.3cm]
% }}}




\resheading{\textbf{\large Major Projects and Seminar}}\\[\lsep] \\[-0.4cm]
% {{{
\begin{itemize}

 \item \textbf{Wikipedia Text Categorization using Multi-Instance Multi-Label (MIML) Classification} \hfill (May'17 - Ongoing)\\
    \emph{M.Tech. Project, Guide: Prof. Ganesh Ramakrishnan, Co-Guides: P. Jain (MSR, Bengaluru), P. Kar (Asst. Prof., IIT Kanpur)} \\[\hsep]
        \begin{itemize}
        	
\item Designed a model to tag Wikipedia documents with relevant subset of categories from a set of millions of categories (Extreme MIML classification).  \\[\hsep]
\item Converted standard Extreme Single-Instance Multi-Label Wiki10 dataset into Extreme MIML dataset.\\[\hsep]
\item Applied deep neural network architectures for Wikipedia document categorization in Extreme MIML setting. \\[\hsep]
\item Future Work involves developing a novel approach to generate semantic paragraph embeddings for sections of Wikipedia pages. 
        \end{itemize}
\vspace{-0.3cm}
 %}}

%{{

 \item \textbf{Extreme Multi-Label Classification and Performance Measures} \hfill(Spring 2017)\\
    \emph{M.Tech. Seminar, Guide: Prof. Ganesh Ramakrishnan} \\[\hsep]
        \begin{itemize}
	        \item Reviewed state of the art Extreme Multi-Label Classification methods like Multi-label Random Forests, SLEEC classifier, FastXML and PFastreXML, and associated algorithms such as Perceptron@k .  \\[\hsep]
            \item Conducted literature survey in MIML performance measures such as Prec@k, Recall, F-score etc. 
        \end{itemize} 
\vspace{-0.3cm}
 %}}
  
%         %{{
%          \item \textbf{Telstra Network Disruption Prediction  } \hfill(Autumn 2016)\\
%     \emph{Course Project, CS 725: Machine Learning} \\[\hsep]
%         \begin{itemize}
% 	        \item Attempted Kaggle Competition to predict Telstra Network's fault Severity at a time and at a particular location based on the log data available. \\[\hsep]
%             \item Applied Machine learning ensembling models- Random Forest, XGBoost and Extra Trees Classifier.    
%         \end{itemize} 
%       \vspace{-0.3cm}
%           %}} 
          
          
%{{

\item \textbf{Simulation of Routing Protocol for MANET Environment} \hfill (Jun'14 - Apr'15) \\
    \emph{B.Tech. Project, Guide: Mr. Dhiraj Pandey } \\[\hsep]
        \begin{itemize}
            \item Implemented DLAR (Direction and Location Aided Routing) routing protocol which incorporates directional capabilities to find the most optimal route between two communicating stations. \\[\hsep]
            \item Designed an algorithm for optimized route request which reduces the number of control packets sent. \\[\hsep]
            \item \textbf{Language and Tools Used}: C++, NS 2.32 Simulator and NAM 1.14	\\[\sectsep]
            \end{itemize}
       %}}
\end{itemize}		

% }}}




%{{{
%for table
\setlength{\tabcolsep}{3.5pt}
\renewcommand{\arraystretch}{1.3}

\resheading{\textbf{\large Work Experience}}\\[\lsep] \\[-0.4cm]
\begin{itemize}
\item \textbf{  Reve Systems India Pvt. Ltd. \emph{(Engineer - Software Development)} }  \hfill  (Jun'15-Jun'16) \\
	 \emph{Project: WebRTCGateway }	 \\ [\hsep]
     \begin{itemize}
     \item  WebRTC Gateway connects WebRTC clients to an established VoIP technology such as SIP. \\[\hsep]
     \item It anchors signaling and media, and performs translation between different standards for WebRTC and SIP.\vspace*{-5mm}
     \end{itemize}

    \end{itemize}
    
    
     \begin{table}[h!]\centering 
          
            \begin{tabular}{ c  c c  c } 
            \toprule[0.01pt]
           % \rowcolor{Gray} 
            \textbf{Modules Developed} & \textbf{Description} &   	\textbf{Language}   & 	\textbf{Communication Protocol}  \\ \midrule[0.01pt]
            
       \rowcolor{Gray}      \textbf{Authentication Server} & \multicolumn{1}{m{6.4cm}}{It acts as an Authenticator as well as a Provisioning server.} &  C/C++ &  HTTPS, SSL \\
            
            \textbf{Service} &  \multicolumn{1}{m{6.4cm}}{ It acts as a bridge between Database and other Gateway components.} & Java & SSL, TCP \\  
            
      \rowcolor{Gray}       \textbf{Client side API} & \multicolumn{1}{m{6.4cm}}{Wrapper API over JsSIP SIP stack. \newline Modified and added modules in JsSIP \newline library to fulfill requirements.} & 	Javascript 	&	 HTTPS, Websockets \\ \bottomrule[0.01pt]
            
     \end{tabular} 
     \end{table}    \vspace{-0.2cm}

%}}}


%{{{

\resheading{\textbf{\large Course Projects}}\\[\lsep]
\\[-0.4cm]

\begin{itemize}

       
%{{
         \item \textbf{Telstra Network Disruption Prediction  } \hfill\emph{(CS725: Machine Learning, Autumn 2016)}  \\[\hsep]
        \begin{itemize}
	        \item Implemented a model to predict Telstra Network's fault Severity at a time and at a particular location based on the log data available (Kaggle Competition). \vspace{-0.1cm}
            \item Applied Data Preprocessing and Decision Tree based models: Random Forest and XGBoost in Python.  
        \end{itemize} 
      \vspace{-0.3cm}
          %}} 

     \newpage  
          
% %{{
%          \item \textbf{Machine Learning Implementations } \hfill\emph{(CS725: Machine Learning, Autumn 2016)}  \\[\hsep]
%         \begin{itemize}
% 	        \item Predicted popularity of online news, using Ridge Regression in Python. 
%             \item Implemented Neural Network from scratch with backpropogation in Python and performed parameter tuning to predict the Poker hand for a given cards.  
%         \end{itemize} 
%       \vspace{-0.3cm}
%           %}} 
	
%{{
    \item \textbf{Calculus of Communicating Systems to State Machine Converter} \hfill \emph{(CS718: Software Architecture, Spring 2017)}	\\[\hsep]
        \begin{itemize}
        	\item Designed architecture of an application that converts given Calculus of Communicating Systems (CCS) expression to sate machine using activity diagram, class diagram and state chart diagram.
            \item Developed a C++ application that achieves this via storing intermediate state machine representations. 
        \end{itemize}
        \vspace{-0.3cm}
       %}} 
       

	%{{
    \item \textbf{Socket Programming based File Sharing System} \hfill\emph{(CS641: Computer Networks, Autumn 2016)} \\[\hsep]
        \begin{itemize}
	        \item Designed and Implemented a modular architecture of peer to peer file sharing system for publishing, searching, uploading and downloading of files from a peer. 
            \item \textbf{Language and Communication Protocol Used}: C/C++, TCP 
        \end{itemize}
        \vspace{-0.3cm}
      %}}  


%{{
    \item \textbf{Shared memory based Inter-VM (application) communication } \hfill\emph{(CS695: Virtualization, Autumn 2016)}  \\[\hsep]
        \begin{itemize}
	        \item Built a wrapper API over IVSHMEM library to provide a shared memory based interface for communication between guest OSes and between a guest OS and the host.
            \item \textbf{Language and Technologies Used}: C/C++, Qemu/KVM hypervisor 
        \end{itemize}
         \vspace{-0.3cm}
      
        %}} 

%{{
    \item \textbf{BuySell Market: Online platform for buying and selling products} \hfill\emph{(CS699: Software Lab, Autumn 2016)}  \\[\hsep]
        \begin{itemize}
	        \item Developed a web application using HTML, PHP and MySQL for easy buying and selling of products online.
            \item Prospective buyers can browse products and contact seller, sellers can put their product on sale. \\[\sectsep] 
        \end{itemize}
         
      
        %}} 

\end{itemize}
%}}}





%{{{
\resheading{\textbf{\large Scholastic Achievements}} \\[\lsep] \\[-0.4cm]
\begin{itemize}
\item \noindent Awarded \textbf{Department Rank-15} among 100 students  in M.Tech Computer Science and Engineering.  \hfill (2016-17) \\[\hsep]

\item \noindent Secured \textbf{AIR-17} amongst 1,08,495 candidates in \textbf{GATE Computer Science}.  \hfill (2016) \\[\hsep]

\item \noindent Ranked \textbf{1st} in B.Tech among 142 students of same batch in CSE department. \hfill (2015) \\[\hsep]

\item \noindent Secured \textbf{2nd prize} in \textit{Algorythm Event, Technophilia'13}, Intercollegiate Technical Festival, JNU, Delhi.  \hfill (2013) \\[\hsep]

\item \noindent Awarded Certificate of Merit for securing \textbf{Institute Rank 2} in $IV^{th}$ semester in a batch of 936 students and \textbf{Institute Rank 3} in $I^{st}$ semester in a batch of 819 students in B.Tech.	\hfill (2011-13) \\[\hsep]

\item \noindent Granted \textbf{Tuition Fee Waiver Scholarship} from Dr. A.P.J. Abdul Kalam Technical University for B.Tech.	\hfill (2011-15)\\[\hsep]
\item FOSSEE certified for Python workshop completion, conducted by Spoken Tutorial Project, IIT Bombay. \hfill (2012)\\[\sectsep]
\end{itemize}

%}}}


%{{{

\resheading{\textbf{\large Key Courses Taken}}\\[\hsep]
\\[0.1cm]

\noindent\begin{minipage}{.35\textwidth}
		\begin{itemize}
			\item Foundations of Machine Learning\\[\hsep]
			\item Web Search and Mining
		\end{itemize}
	\end{minipage} 
    \noindent\begin{minipage}{.30\textwidth}
    \begin{itemize}
			\item Probabilistic Models\\[\hsep]
            \item Computer Networks
		\end{itemize}
     \end{minipage}   
	\noindent\begin{minipage}{.30\textwidth}
		\begin{itemize}
			\item Algorithms and Complexity\\[\hsep]
			\item Virtualization 
		\end{itemize}
	\end{minipage}
    \vspace{0.2cm}
%}}}




\resheading{\textbf{\large Technical Skills}} \\[\lsep] \\[-0.4cm]
%{{{
\begin{itemize}
\item \noindent \textbf {Programming Languages}: C, C++, Java, Python, Awk, Sed		\\[\hsep]

\item \noindent \textbf{Tools and Technologies}: Tensorflow, \LaTeX, Wireshark, SIP, Git \\ [\sectsep]
\end{itemize}
%}}}



%{{{
\resheading{\textbf{\large Positions of Responsibility}}\\[\lsep] \\[-0.4cm]


%{{

%{
\begin{itemize}
	\item \textbf{Teaching Assistantship, IIT Bombay} \\[\hsep]
      \begin{itemize}
              \item \textbf{CS725 Foundations of Machine Learning}  
              \emph{(under Prof. Ganesh Ramakrishnan)}\hfill (Autumn 2017) \\\vspace{-0.5cm}
          \begin{itemize}
%            \item Worked in a team of 7 members for formulating strategies for better organization of the course. \\[-0.5cm]
%               \item Organized quizzes, prepared assignments, coordinated project evaluations and mentored 28 students in batch of 190 students.
              \item Formulated strategies for better organization of the course, prepared assignments, resolved course related queries, coordinated project evaluations and mentored students in their project preparation.
              
          \end{itemize}
          \vspace{-0.1cm}
       	%}
        
        
        %{ 
          \item \textbf{CS226 Digital Logic Design} \emph{(under Prof. Supratik Chakraborty)}\hfill (Spring 2017) \\\vspace{-0.5cm}
              \begin{itemize}
                   \item Prepared practice problems, organized quizzes and evaluated answer scripts for B.tech students.
              \end{itemize}
               \vspace{-0.1cm}
       	%}
          
          %{
          \item \textbf{CS101 Computer Programming}
              \emph{(under Prof. Bernard Menezes)}\hfill (Autumn 2016) \\\vspace{-0.5cm}
          \begin{itemize}
              \item Assisted students during weekly labs in understanding the concepts of programming and graded programming lab exercises.  \\[\psep]
          \end{itemize}
           \vspace{-0.1cm}
       	%}
          
       \end{itemize}
 %}      
       
       %{{
	\item \textbf{Student Companion, ISCP IIT Bombay} \hfill (Jul'17 - Ongoing) \\[\hsep]
		\begin{itemize}
        	\item \noindent Worked in team of 164 members to organize orientation programmes, course registrations and informal interaction sessions for 1449 PG Freshers.  \\[-0.5cm]
        	\item \noindent Guiding and supporting a group of 5 students on both academic and personal fronts. \\[\psep]
     	\end{itemize} 
         \vspace{-0.3cm}
       	%}}
        
        
        %{{
 
 \item \noindent Coordinated the online event \textbf{Algothematics} in Zealicon, TechFest of JSS Noida and prepared questions based on mathematical problems.  \\[\sectsep]
         %}}
\end{itemize}
%}}}




\resheading{\textbf{\large Extra-Curricular Activities}}\\ [\lsep]
\begin{itemize}
\item CSE representative, Table Tennis for 2 consecutive years in PG General Championship at IIT Bombay. \hfill (2016-17)\\[\hsep]
\item \noindent Represented CSE department in Badminton and Lawn Tennis PG General Championship at IIT Bombay. \hfill (2016-17) \\[\hsep]
\item \noindent Participated in Table Tennis Tournament hosted for over 100 employees from different departments in Reve Systems India Pvt. Ltd. \hfill (2016)\\[\hsep]
\end{itemize}
\end{document}

